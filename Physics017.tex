%% LaTeX Paper Template, Flip Tanedo (flip.tanedo@ucr.edu)
%% GitHub: https://github.com/fliptanedo/paper-template-2022

% \documentclass[11pt]{article} %% Not for Lecture Notes
\documentclass[12pt, oneside]{report}    %% Has chapters

\input{FlipLectureMacros}       %% Lecture note formatting, load first
\input{FlipPreamble}			%% \usepackages, formatting
\input{FlipMacros}              %% Flip's standard macros
\input{FlipMacros_Comments}     %% Flip's macros for comments
\input{FlipMacros_Teaching}     %% Flip's macros for course notes
\input{Flip_listings}           %% Styling for code blocks
\input{FlipAdditionalHeader}    %% Modify this for each project
\input{FlipPreambleEnd}         %% packages that have to be at the end

%%%%%%%%%%%%%%%%%%%%%%%%%%%%
%% LECTURE NOTES SETTINGS %%
%%%%%%%%%%%%%%%%%%%%%%%%%%%%

% \linenumbers                  %% print line numbers (lineno package)
\graphicspath{{figures/}}       %% figure folder
\addbibresource{FlipBib.bib}    %% Define BibLaTeX source(s)

%% LEAVE THESE HERE 

\geometry{                      %% large margin for side notes
    paper=letterpaper, 
    hmargin={1cm,7.25cm},       %% 6.25cm space on right
    vmargin={2cm,2cm}, 
    marginparsep=.5cm, 
    marginparwidth=5.75cm
}

%% Def. full width; uses changepage package; 6.25cm to match hmargin difference;
\newenvironment{wide}{\begin{adjustwidth}{0cm}{-6.25cm}}{\end{adjustwidth}}


% Reset the sidenote number each section 
\let\oldsection\section
\def\section{%
  \setcounter{sidenote}{1}%
  \oldsection
}


\begin{document}

\newgeometry{margin=2cm}                   % plain geometry for frontmatter
\newcommand{\FlipTR}{UCR-TR-S2024-FLIP-P017} % TR#, pdfsync may fail on 1st page
\thispagestyle{firststyle} 	               % TR#; otherwise use \thispagestyle{empty}
\pagenumbering{gobble}                     % no page number on first page 

%%%%%%%%%%%%%%%%%%%%%%%%
%%%  FRONTMATTER    %%%%
%%%%%%%%%%%%%%%%%%%%%%%%


\begin{center}
    {\large \textsf{UC Riverside Physics 017, Spring 2024} \par}
    {\huge \textbf{Linear Algebra for Physicists} \par}\vspace{.5em}
    {\large {Tensors, kets, indices, metrics and all that...} \par}
    \vskip .5cm
\end{center}

\input{FlipAuthors}

\vspace{2em}\noindent
Lecture notes for Physics 17, a course on linear algebra in preparation for upper-division undergraduate physics coursework at \acro{UC~R}iverside.

\vspace{5em}
\begin{center}
\includegraphics[width=.2\textwidth]{figures/Squiggle.pdf}
\end{center}  


% \vspace{2em}
\vspace*{\fill}

\noindent
\textsf{Last Compiled: \today}

\noindent
\textsf{Image: Birdtrack notation for tensor contraction}

\noindent
\textsf{CC BY-NC-SA 4.0}~\ccbyncsa 

\noindent % Course notes URL
% \url{https://github.com/fliptanedo/P231-2023-Math-Methods}

%% Front page logos
\vspace*{\fill}
\begin{center}
\includegraphics[height=.1\textwidth]{figures/FlipAmbigram.png}
\hspace{5em}
\includegraphics[height=.1\textwidth]{figures/UCRPnA_banner.png}
\end{center}

\newpage

\small
\setcounter{tocdepth}{2}
\tableofcontents
\normalsize
\clearpage
\restoregeometry        %% Return to lecture note geometry 
\pagenumbering{arabic}  %% Turn on regular page numbers


%%%%%%%%%%%%%%%%%%%%%
%%%  THE CONTENT  %%%
%%%%%%%%%%%%%%%%%%%%%

%% TEMPLATE STUFF
% \input{examples_lecture}
% \chapter{Paper examples}

% Here are the standard examples I use for my \texttt{paper} template. I include them here to check that nothing has broken. These do not make use of the margin at all. You can see what happens when some text spills into the margin unintentionally.

% \input{examples}
% \input{examples_teaching}
% \input{examples_listings}
% \input{examples_bestpractices}
% \input{examples_refs}

% %% CHAPTER SUBAPPENDIX %% if using report class
% \begin{subappendices}
% \section{Subappendix}\label{sec:subappendix:eg}
% This chapter has its own special appendix.
% \end{subappendices}


\chapter{Introduction}

\section{Two powerful questions}
At any time in this course, you should feel comfortable asking either of the following questions:
\begin{enumerate}
    \item Is it obvious that...?
    \item Why is this significant?
\end{enumerate}
The first question is the way to ask for on-the-spot clarification---I will either appreciate that I did not properly explain a subtle point, \emph{or} I will explain the intuition\sidenote{Your sense of mathematical and physical intuition is incredibly valuable. This is one of the key traits that makes a physics training unique.} for why something should be obvious. The second question is a way to remind me that I may have \emph{lost sight of the forest for the trees}: I want this course to \emph{mathematically connect big ideas in physics}. Asking this question is a reminder that making those connections justifies the hard mathematical work we will put into the course.

\section{Exercises and Examples}
I have tried to insert exercises and examples in these notes. There are still far too few for sound pedagogy. If you really, really want to learn something, you \emph{have} to do exercises. Think of the examples as exercises with solutions---though they are not always written this way. Mull over the exercises: ask yourself why the problems are posed the way they are, challenge the statements to find the domain of validity, think of how one may extend those exercises to other applications. The exercises are a far better gauge of you learning than whether or not you have read a section of the notes. If you are confused reading the text in section 10, it is often the case that you should have been doing the exercises since section 5.

\begin{bigidea}[Do your homework]
Instructors feel no deep satisfaction when you turn in your homework. Instead, an assignment is a pledge to the student to give an opportunity for practice with feedback from someone more experienced.
\end{bigidea}

\section{Obvious-ness}
Finally, I want to comment on the word \emph{obvious}. I write this often. It is somewhat dangerous because it can come off as being arrogant: \emph{this is so obvious to me, if you do not understand you must be deficient}. This is never the reason why I use that word. Instead, the word \emph{obvious} serves a very practical purpose. The goal of this class is not just to be able to ``do stuff'' (e.g.~diagonalize a symmetric matrix), but to also build that intuition that comes from a deeper understanding how the mathematics works. In this sense, every time I write the word \emph{obvious} it is a flag: I am saying something that---with the proper perspective---should be self-evident. If it is not self-evident, then you should stop to interrogate why it is not self-evident. Most likely there is something where a change in perspective may (1) make it obvious, and (2) in so doing deepen your understanding of the subject. So when you see the word `obvious,' I want you to do a quick check to confirm whether or not the statement is indeed obvious. If it is not, then welcome the opportunity to learn.\sidenote[][-3em]{There is, of course, the possibility that what I have written is \emph{not} obvious. For example, if I have made a typo... in which case, please let me know.}



\section{Motivation}

Here are three deeply significant equations in physics:\sidenote{If you want to be fancy, you can add Maxwell's equations. If you want to be \emph{really} fancy, you can write these as $dF=0$ and $-*dF = *J$, but that's for a different course.}
\begin{align}
    \vec{F} &= m\vec{a}
    \\
    % R_{\mu\nu} - \frac{1}{2}Rg_{\mu\nu} 
    G_{\mu\nu}
    &= \frac{8\pi G_\text{N}}{c^4} T_{\mu\nu}
    \\
    \hat H |\Psi\rangle 
    &= E |\Psi\rangle \ .
    \label{eq:three:equations}
\end{align}
These are Newton's force law, Einstein's field equations, and the Schr\"odinger equation. They govern classical physics, general relativity, and quantum theory, respectively. 


Each equation looks rather unique: they seem to each be speaking their own mathematical language. Newton's law is written with boldfaced vectorial quantities $\vec{F} = (F_x, F_y, F_z)^\trans$ that should look very familiar to any physics undergraduate. Einstein's equation has these funny $\mu$ and $\nu$ indices on every term---have you seen these before? Do they look intimidating? If you ever want to make your equations look ``technical'' and ``physicsy,'' you should dress them up with indices. The Schr\"odinger equation has no indices, but instead has these funny angle-brackety things... and that $\hat H$ looks suspicious. Where did $H$ get a hat, and what is the content of this equation other than $\hat H = E$?

\emph{Each of these equations turns out to be a ``vectorial'' equation.} Each one is actually shorthand for a number of equations. Newton's equation is shorthand for three equations, one for each component. Einstein's equation is shorthand for 16 equations, one for each combination of the indices $\mu$ and $\nu$ that run over four values\sidenote{The four values are the three directions of space and one direction of time.}. The Schr\"odinger equation is shorthand for an \emph{infinite} number of equations, one for each allowed energy of a quantum system.

The mathematical formalism that unifies these different ideas (and notations) of `vector' is called linear algebra. It may sound humble: after all, ``linear'' systems are \emph{easy}, aren't they? Did we not just spend years of our lives learning fancy things like \emph{calculus} and \emph{differential equations} to deal with functions that are more complicated than \emph{lines}? In some sense, yes: linear algebra is about lines and planes in different numbers of dimensions.\sidenote{On the other hand: a good chunk of the calculus that we do is also implicitly linear. Physicists often Taylor expand and keep only the $\mathcal O(\varepsilon)$ term. Integration boils down to summing trapezoids whose angley-bits are given by the first derivative of a function... the linear component.} However, linear algebra turns out to be far more richer than what you may be used to from high school. 

In this course we will see how the three equations in \eqref{eq:three:equations} are connected by the mathematics of linear algebra. We will dive into the different notation and shamelessly pass between $\vec{v}$, $v^i$, and $\ket{v}$ to describe the same abstract vector. We will connect to the mathematical description of \emph{symmetry} and see how it is an underlying theme in our descriptions of nature. And we will do all of this in a way that will make the instructors of the linear-algebra-for-mathematicians course and linear-algebra-for-engineers course vomit a little in disgust. Consider that one of privileges of being a physicist.



\chapter{The Basics}

\section{Pre-conceptions}

If this were a mathematics course, then we would start by very carefully defining words like \emph{vector} and \emph{matrix}. As a physics student, you already have a working definition of these words. It is probably something like this:
%
\begin{quote}
A vector has a magnitude and a direction. We write a vector as an array of three numbers arranged in a column. A matrix is an array of nine numbers arranged in a $3\times 3$ block. There is a rule for how to apply (multiply) the matrix to the vector to produce a new vector.
\end{quote}

The problem is that you already know too much to learn linear algebra as a mathematics student. You have already seen the tip of the iceberg and so have preconceptions about what vectors are and how they work. You may remember from freshman mechanics that forces are vectors. So are momenta and velocities. You may also recall the idea of a force field---like the electric field---which is actually a whole bunch of vectors: one for each point in space. Examples of matrices are a little more subtle: you may recall that you can represent rotations as matrices. Speaking of rotations, there was another thing that showed up called the moment of inertia \emph{tensor}. It looked like a matrix, but we never called it the ``moment of inertia matrix.'' What the heck is a tensor, anyway?

And so, you see that starting this course like a mathematics course could cause trouble. The mathematics professor would start by defining a vector. That definition will say nothing about magnitudes or directions, and will not even say anything about arrays of numbers. That definition will clash with the hard-earned intuition that you built from your physics education thus far. It will be perplexing, and may make you feel rather unhappy. What do these mathematicians know, anyway? Or maybe its the physics that is wrong, or have we just completely misunderstood everything and we are just now noticing that we are hopelessly lost? We begin to spiral into a black hole of confusion.

\begin{quote}
Fortunately, \emph{this is not a mathematics course.}
\end{quote}

As a consequence, we will not give a rigorous definition of a vector. We start with a familiar definition of vectors and lay out which qualities are general, and which properties are specific. Then we will come to appreciate the approximation that ``\emph{everything is a vector}.'' So let us start with something comfortably familiar, even though it constitutes only the simplest example of a vector.


\section{Real Three-Vectors}

Let us write $\vec{v}$ to be a vector. This is a standard convention for writing a vector. In this course we will use a few different notations for vectors according to convenience. Notation is neither physics nor mathematics, it is simply a shorthand for a physical or mathematical idea. 

% At this point, you may wonder \emph{what is a vector, anyway?} Maybe a vector is a column with three numbers that represent coordinates in three-dimensional space:
In fact, let us focus on a particular type of vector: \textbf{real three vectors}\index{three vector}. These are the familiar vectors that we can write as a column of three numbers that effectively represent the coordinates in three-dimensional space:
\begin{align}
    \vec{v} = 
    \begin{pmatrix}
        x\\ y\\ z
    \end{pmatrix} \ ,
\end{align}
where $x$, $y$, and $z$ are real numbers. These numbers are called the \textbf{components} of the vector $\vec{v}$.

\begin{exercise}
There is something very perverse about this ``vector.'' The variable names $x$, $y$, and $z$ imply that $\vec{v}$ is something that physicists like to call a ``position vector.'' If you say this to a mathematician they will vomit. By the end of this course, you should appreciate why the notion of a position vector makes no sense. \emph{Hint:} You may have some intuition for this already: a velocity vector tells you about the instantaneous motion of a particle relative to its present position. Try to write the analogous statement for a ``position vector.\footnote{I am not a mathematician, but you see that even I have to write ``position vector'' in condescending quotation marks. In lecture I use even more condescending air quotes.}''
\label{ex:position:vector}
\end{exercise}

This three-dimensional space is called [three-dimensional] \textbf{real space} and we write it as $\RR ^3$. This is because a vector is an element of three-dimensional real space specified by \emph{three} real numbers. 

Three-dimensional real space is an example of a \textbf{vector space}\index{vector space}, which is just a stupidly formal way of saying that it is where vectors live. Vectors are \emph{elements} of a vector space. A vector space is the set of all possible allowed vectors of a given type. For $\RR ^3$, the vector space is composed of all possible triplets of real numbers. 


\begin{example} It should be no surprise that we can imagine real two-dimensional space, $\RR ^2$. This is a vector space where each vector may be written as two real numbers. You can also imagine writing real four-dimensional space, $\RR ^2$, or complex two dimensional space, $\mathbbm{C}^2$. 
\end{example}

From the above example, you should have some intuition for what the \textbf{dimension}\index{dimension} of a vector space means: the dimension counts how many numbers you need to specify a vector. For real vector spaces, $\RR ^d$, the dimension is the number $d$. We will always assume that $d$ is a positive integer.\sidenote{The notion of a non-integer-dimensional space does show up occasionally. These do not even have to be particularly exotic: you can look up the dimension of a fractal.}





\section{Vectors and Numbers}

% We now make some general statements about vector spaces. These apply to all vector spaces, not just $\RR ^3$, but you can keep $\RR ^3$ in mind as we go over them. 

We should be clear that there are now two different kinds of objects: \emph{vectors} and \emph{numbers}. We will have all sorts of notation for vectors, but let us write them with a boldfaced Roman letter for now, e.g.~$\vec{v}$. We typically write numbers as lowercase italicized Roman letters like $a$ or sometimes Greek letters like $\alpha$. These two types of objects are similar, except vectors do not have a built-in definition for multiplication, see Table~\ref{table:vectors:numbers}.

\begin{table}
    \renewcommand{\arraystretch}{1.3} % spacing between rows
    \centering
    \begin{tabular}{ @{} lll @{} } \toprule % @{} removes space
         & Vectors & Numbers 
        \\ \hline
        Addition (commutative, associative) & \cmark & \cmark 
        \\
        Additive null element & $\vec{0}$ & 0
        \\
        Additive inverse element & $\vec{v} + (-\vec{v}) = 0$ & $a + (-a) = 0$
        \\
        Multiplication of two of these objects & \textcolor{red}{\xmark} & \cmark 
        \\
        Multiplication by a number (distributive) & \cmark & \cmark \,(same as above)
        \\
        Collection of all allowed objects (space) & vector space & field (``numbers'') 
        \\
        Example of a space & $\RR ^3$ & $\RR $
        \\ \bottomrule
    \end{tabular}
    \caption{
        What you can do with vectors compared to numbers. The glaring difference is that we cannot multiply two vectors. We will need to invent additional mathematical structure to define vector multiplication.
        \label{table:vectors:numbers}
  }
\end{table}

You already know everything there is to know about numbers.\sidenote{Formally, what I mean by `number' is what mathematicians call a \emph{field}. This simply means some objects where one can add, subtract, multiply, and divide as you would expect. This term is a little tedious for us because physicists usually mean something else when they say `field.' Usually we mean something like the electric field or the field associated with the Higgs boson.} Most relevant is that you can multiply numbers with each other (including division, the inverse of multiplication) and you can add them together (including subtraction). For the first part of this course, we will focus on real numbers, $\RR $. Later we will also allow for complex numbers, $\mathbbm{C}$. 

Like numbers, vectors can be added and subtracted. In fact, vector arithmetic turns out to be very similar to `number arithmetic.' However, unlike numbers, there is no obvious definition for vector multiplication. This leads to the idea of \emph{defining} functions for various kinds of vector multiplication. Linear algebra is the study of a particular class of these functions. The dot product, for example, which takes two vectors and returns a number, is something we have to ``make up'' and attach to a vector space.



\section{Notation: Indices}

One theme in this course is that we will repeatedly refine our notation to suit our needs. Let us introduce an \emph{index} notation where we write the components of vectors $\vec{v}$ and $\vec{w}$ as follows:
\begin{align}
    \vec{v}
    &=
    \begin{pmatrix}
        v^1 \\ v^2 \\ v^3
    \end{pmatrix}
    &
    \vec{w}
    &=
    \begin{pmatrix}
        w^1 \\ w^2 \\ w^3
    \end{pmatrix} \ .
\end{align}
We see that a boldfaced Roman letter, $u$, corresponds to a vector. The \emph{components} of the vector are $u^1$, $u^2$, $u^3$. The ``$x$-component'' of $\vec{u}$ is called $u^1$: we use the same letter as the vector, but italicized rather than boldfaced. The upper index is \emph{not} some kind of power, it simply means ``the first component.'' 

\begin{example}
If you see $\vec{s}$, this is understood to be a vector that has multiple components. If it is a three-vector, it has three components. If you see $s^2$, then this means that this is the \emph{second component} of the vector $\vec{s}$. The component of a vector is a number. 
\end{example}

You may worry that this notation introduces ambiguity. If we see $q^2$, is this the square of some number $q$, or is it the second component of some vector $\vec{q}$? The answer depends on context. You should avoid choosing variable names where there is ever the potential for ambiguity. If you have a vector that you call $\vec{q}$, then do not use the letter $q$ for anything else.

\sidenotetext{It is a personal pet peeve of mine that some first year courses for physics majors are sloppy about this. The instructors should know that they are building the foundation for understanding quantum mechanics and special relativity: they should start developing \emph{good} mathematical habits early on. }
\begin{example}
Some courses\sidenotemark write vectors as rows: $\vec{v}=\begin{pmatrix}
    v^1 & v^2 & v^3
\end{pmatrix}$ \ . Even more annoying to me, they may write everything with a lower index, $\vec{v}=\begin{pmatrix}
    v_1 & v_2 & v_3
\end{pmatrix}$ \ . There is nothing \emph{wrong} with this, certainly in classes where there is only one type of vector. However, in order to make full use of linear algebra, we need to treat so-called column vectors separately from so-called row vectors. To do this, we introduce a new set of notation where the heights of the indices are important. Eventually we will get rid of the notion of rows versus columns altogether---but the notion of upper and lower index will remain.
\end{example}


You know from $\RR ^3$ that you can add together any two vectors $\vec{v}$ and $\vec{w}$.
% 
Let us call this sum $\vec{u}$ so that $\vec{u}\equiv \vec{v}+\vec{w}$. Then we can succinctly write the components of $\vec{u}$ in one line:
\begin{align}
    u^i = v^i + w^i \ .
    \label{eq:u:v:plus:w:index}
\end{align}
The variable $i$ is called an \textbf{index}\index{index}. What values does the index take? In this example, it is 
clear that \eqref{eq:u:v:plus:w:index} holds for $i=1,2,3$. That is, $i$ takes values from 1 to the dimension of the space. The typical convention is that we do not have to state the range of index values because it should be understood from the space itself. 

With that in mind, it should be clear that if $\vec{q}$ is the difference of two vectors, then the components of $\vec{q}$ may be succinctly written:
\begin{align}
\vec{q} &= \vec{v}-\vec{w}    
&
&\Leftrightarrow
&
q^i &= v^i - w^i \ .
\end{align}
In fact, as physicists we typically use the two statements above interchangeably. If you know the components of a vector, then you know the vector.


\section{Arithmetic and linear combinations}

All vector spaces allow addition and subtraction. This is defined component-wise. The sum of $\vec{v}$ and $\vec{w}$ is
\begin{align}
    \vec{v}+\vec{w} = 
    \begin{pmatrix}
        v^1 + w^1\\
        v^2 + w^2\\
        v^3 + w^3
    \end{pmatrix} \ .
\end{align}
What this means is that the \emph{sum} of two vectors is also a vector. That means that if $\vec{v}$ and $\vec{w}$ are vectors in $\RR ^3$, then $(\vec{v}+\vec{w})$ is a vector in $\RR ^3$. The components of the vector $(\vec{v}+\vec{w})$ are simply the sum of the components of $\vec{v}$ and $\vec{w}$. 
% 
A few formal properties that generalize to all vector spaces:
\begin{itemize}
    \item Vector addition is associative. This means that in the sum $\vec{v}+\vec{w}+\vec{u}$, it does not matter if you add $(\vec{v}+\vec{w})$ first and then add $\vec{u}$, or if you take $\vec{v}$ and then add it to $(\vec{w}+\vec{u})$. This is the kind of `obvious' property that we tend to take for granted.
    \item Vector addition is commutative. $\vec{v}+\vec{w} = \vec{w}+\vec{v}$. This is also kind of obvious. But recall that matrix multiplication is not commutative.
    \item There is a zero vector, $\vec{0}$, that does leaves any other vector unchanged under addition. $\vec{v}+\vec{0} = \vec{v}$. This should be totally obvious. The components of $\vec{0}$ are obviously all zero.
    \item There is an additive inverse (negative vectors). If $\vec{v}$ is a vector, then $-\vec{v}$ is a vector and satisfies $\vec{v}+(-\vec{v}) = \vec{0}$.
\end{itemize}
\begin{example}
The first property implies that once you have identified one vector in a vector space, $\vec{v}$, then you can immediately have an infinite number of vectors. This is because $2\vec{v} = \vec{v}+\vec{v}$ must also be a vector. Then $3\vec{v} = 2\vec{v}+\vec{v}$ must also be a vector. And so forth.
\end{example}

We get another type of operation ``for free'' with a vector space. This is called scalar multiplication or \emph{rescaling}.

% You can multiply vectors by numbers. This is called rescaling or scalar multiplication. All the usual properties of multiplication by numbers holds: associativity, commutivity, distributive law.


\section{Rescaling: multiplication by a number}

Another operation that exists in a vector space is rescaling: we multiply a vector by a number. 
Let $\alpha$ be a number. If you want to nitpick, let us restrict $\alpha$ to be a real number. If we have a vector $\vec{v}$ with components $v^i$, then $\alpha \vec{v}$ is also a vector.\sidenote{``Also a vector'' means that it is also an element of the vector space; so $(\alpha\vec{v})$ is an element of $\RR ^3$ is $\vec{v}$ is an element of $\RR ^3$. } The components of $\alpha \vec{v}$ are
\begin{align}
    (\alpha v)^i = \alpha v^i \ ,
\end{align}
by which we mean
\begin{align}
    (\alpha\vec{v})
    =
    \begin{pmatrix}
        \alpha v^1 \\
        \alpha v^2 \\
        \alpha v^3 
    \end{pmatrix} \ .
\end{align}
The parenthesis on the left-hand side is sloppy notation to mean ``the vector that is the vector $\vec{v}$ rescaled by the number  $\alpha$.'' Another way of saying this is that there is a vector $\vec{w}\equiv \alpha\vec{v}$ whose components are $w^i = \alpha v^i$.

\begin{example}
Let us do one explicit example with numbers. Suppose the vectors $\vec{v}$ and $\vec{w}$ have components
\begin{align}
    \vec{v} &=
    \begin{pmatrix}
    \phantom{+}4.2\\
    -2.6\\
    \phantom{+}7.0        
    \end{pmatrix}
    &
    \vec{w} &=
    \begin{pmatrix}
    \phantom{+}5.3\\
    \phantom{+}2.1\\
    -2.5        
    \end{pmatrix} \ .
\end{align}
I can rescale each vector by different numbers: $\alpha = 10$, $\beta = 2$. We can consider the vector that comes from adding these rescaled vectors:
\begin{align}
    \vec{u} \equiv \alpha \vec{v} + \beta \vec{w} \ .
\end{align}
The second component of $\vec{u}$ is $u^2 = -26 + 4.2 = -21.8$.
\end{example}

At this point it is useful to define some jargon. A \textbf{scalar}\index{scalar} is a number. This is in contrast to vectors (and matrices and tensors) which we can think of as arrays of numbers. In fact, every time you see the word scalar, you should just think ``number.'' Another name for `rescaling a vector by a number' is \emph{scalar multiplication}.

\section{Linear Combination and Span}

Based on our rules for vector space arithmetic, we know that if $\vec{v}$ and $\vec{w}$ are two vectors in our vector space and if $\alpha$ and $\beta$ are any two numbers, then
\begin{align}
    \alpha\vec{v} + \beta\vec{w} 
\end{align}
is also a vector in our vector space. We call any such sum---for any values of $\alpha$ and $\beta$---a \textbf{linear combination}\index{linear combination} of the vectors $\vec{v}$ and $\vec{w}$. You can of course generalize to the linear combination of more than two vectors, say
\begin{align}
    \alpha\vec{v} + \beta\vec{w} + \gamma\vec{u} \ .
\end{align}

Given some number of vectors---$\vec{v}$ and $\vec{w}$---you can ask what are all of the possible vectors that you can form from the linear combination of those vectors? This is a vector space.\sidenote{You may want to convince yourself that this satisfies the requirements of vector space arithmetic.} We say that this vector space is \textbf{spanned} by the vectors $\vec{v}$ and $\vec{w}$. We call this vector space $\text{Span}(\vec{v},\vec{w})$. You can extend this to even more vectors, $\text{Span}(\vec{v}, \vec{w}, \vec{u},\cdots)$.

\begin{exercise}
Show that the vector space spanned by $\vec{v}$ and $\alpha\vec{v}$ is the same as the vector space spanned by $\vec{v}$.
\end{exercise}

\begin{exercise}
If $\RR ^3$ is the space of vectors with three real components, argue that the span of any four vectors is at most $\RR ^3$ but possibly a subset of $\RR ^3$. Give an example where the span of four vectors is $\RR ^2$. 
\end{exercise}


\section{Basis vectors: an illustrative example}

Let us push this idea further. It is useful to start with an example. For simplicity, let us focus on the two-dimensional plane, $\RR ^2$. A vector in $\RR ^2$ looks like this:
\begin{align}
    \vec{v} =
    \begin{pmatrix}
        v^1 \\ v^2
    \end{pmatrix} \ .
    \label{eq:v:v1:v2}
\end{align}
Any such vector may be written as the linear combination of the following two vectors:
\begin{align}
    {\bas{e}}_1 &=
    \begin{pmatrix}
        1 \\ 0
    \end{pmatrix}
    &
    {\bas{e}}_2 &=
    \begin{pmatrix}
        0 \\ 1
    \end{pmatrix} \ .
\end{align}
Indeed, it should be obvious that 
\begin{align}
    \vec{v} &= \alpha {\bas{e}}_1 + \beta {\bas{e}}_2
    & \text{with}&
    &\alpha &= v^1
    &\beta &= v^2 \ .
    \label{eq:natural:cartesian:basis}
\end{align}
In other words, these `special' vectors ${\bas{e}}_{1,2}$ satisfy:
\begin{enumerate}
    \item Any vector in $\RR ^2$ may be written as a linear combinations of ${\bas{e}}_{1,2}$. We showed this because $\vec{v}$ in the above discussion could be any vector in $\RR ^2$. Thus $\text{Span}({\bas{e}}_{1},{\bas{e}}_{1})=\RR ^2$.
    \item The coefficients in the linear combination are precisely the components of the vector $\vec{v}$. Soon we will see that this observation is actually backwards: it is the choice that ${\bas{e}}_{1}$ are special that defines the components of a vector.
\end{enumerate}

It should be obvious that any pair of vectors that ``aren't pointing in the same direction'' can span the entire space $\RR ^2$. 
\begin{exercise}
What does ``aren't pointing in the same direction'' mean in this context? Use $\vec{v}$ and $\alpha\vec{v}$ in your answer.
\end{exercise}
We could try a different pair of vectors and consider its linear combinations:
\begin{align}
    \vec{f}_1 &=
    \begin{pmatrix}
        1\\1
    \end{pmatrix}
    &
    \vec{f}_2 &=
    \begin{pmatrix}
        0\\1
    \end{pmatrix} \ .
\end{align}
Then the vector $\vec{v}$ may be written as $\vec{v} = \alpha \vec{f}_1+ \beta\vec{f}_2$. To find $\alpha$ and $\beta$, we can simply plug in the components of $\vec{v}$ and $\vec{f}_{1,2}$ so that:
\begin{align}
    v^1 &= \alpha -\beta
    &
    v^2 &= \alpha \ .
\end{align}
In other words,
\begin{align}
    \vec{v} = (v^2) \vec{f}_1 + (v^2 - v^1)\vec{f}_2 \ .
\end{align}
These coefficients $\alpha = v^2$ and $\beta = v^2 - v^1$ may be written  in shorthand. Let's suggestively write
\begin{align}
    \vec{v} = 
    \begin{pmatrix}
        v^2\\
        v^2 - v^1
    \end{pmatrix}_{\vec{f}} \ ,
\end{align}
where we use the subscript $\vec{f}$ to mean ``coefficients with respect to $\vec{f}_{1,2}$. This looks just like a two-component vector, doesn't it?

\begin{exercise}
Let $\vec{v} \in \RR ^2$ be the following vector in two-dimensional real space:
\begin{align}
    \vec{v}=
    \begin{pmatrix}
        3\\2
    \end{pmatrix} \ .
\end{align}
Here are two vectors that span $\RR ^2$:
\begin{align}
    \vec{g}_1 &=
    \begin{pmatrix}
        2\\1
    \end{pmatrix}
    &
    \vec{g}_2 &=
    \begin{pmatrix}
        -1\\ \pp 0
    \end{pmatrix} \ .
\end{align}
What are the coefficients $\alpha$ and $\beta$ so that $\vec{v} = \alpha \vec{g}_1 + \beta \vec{g}_2$?  \emph{Answer}: $\alpha = 2$ and $\beta = 1$. 
\end{exercise}


What we're getting at is the following.  Define a set of vectors that span a space. We will call this set of vectors a \textbf{basis}\index{basis} of that space---we'll give a slightly more formal definition below. Any vector in the space can be written as a linear combination of basis vectors. For example, if $\vec{b}_{1,2}$ are a basis of $\RR ^2$, then for any vector $\vec{v}\in\RR ^2$ we may write
\begin{align}
    \vec{v} = \alpha \vec{b}_{1} + \beta \vec{b}_2 \ .
\end{align}
Then we have encoded all of the data of vector $\vec{v}$ into the coefficients $(\alpha, \beta)$. In fact, let me be more economical with my symbols and change notation a bit and write $(\alpha^1, \alpha^2) \equiv (\alpha,\beta)$ so that
\begin{align}
    \vec{v} = \alpha^1 \vec{b}_{1} + \alpha^2 \vec{b}_2 \ .
    \label{eq:R2:vec:v:in:b:components:lincomb}
\end{align}
Then I can write the information encoded in $\vec{v}$ as a column, which I will write with a subscript $b$ to distinguish it from the ``actual'' vector components, \eqref{eq:v:v1:v2}:\sidenote{We will soon see that the there is nothing holy about \eqref{eq:v:v1:v2}.}
\begin{align}
    \vec{v} = 
    \begin{pmatrix}
        \alpha^1\\
        \alpha^2
    \end{pmatrix}_{\vec{b}} \ .
    \label{eq:R2:vec:v:in:b:components}
\end{align}
The last two equations mean exactly the same thing. Now here's something cute: we can treat the two-component array\sidenote{I'm trying not to call it a vector.} on the right-hand side of \eqref{eq:R2:vec:v:in:b:components} as if it were a vector. We can do vector arithmetic on it. If we have two vectors with ``$\vec{b}$ basis components''
\begin{align}
    \vec{v}&=
    \begin{pmatrix}
        \alpha^1\\
        \alpha^2
    \end{pmatrix}_{\vec{b}} 
    &
    \vec{w}&=
    \begin{pmatrix}
        \beta^1\\
        \beta^2
    \end{pmatrix}_{\vec{b}}  \ ,
\end{align}
Then we could take linear combinations of the two with respect to two numbers $a$ and $b$:
\begin{align}
    a\vec{v} + b\vec{w} =
    (\alpha^1+\beta^1) \vec{b}_{1} + (\alpha^2+\beta^2) \vec{b}_2
    =
    \begin{pmatrix}
        \alpha^1 + \beta^1 \\
        \alpha^2 + \beta^2
    \end{pmatrix}_{\vec{b}} \ .
    \label{eq:linear:combination:in:b:basis}
\end{align}

If $\vec{v}$ and $\vec{w}$ span $\RR ^2$, then any vector in the vector space may be written as a linear combination of the form \eqref{eq:linear:combination:in:b:basis}. This means we may use the ``$\vec{b}$ basis components'' as equivalent ways of encoding a vector as the natural description \eqref{eq:v:v1:v2}. But wait a moment: what is so ``natural'' about \eqref{eq:v:v1:v2}? 

If we reverse the argument for the $\vec{b}$ basis, then we see that the ``natural'' components of the vector $\vec{v}$ in \eqref{eq:v:v1:v2} are simply the coefficients of the linear combinations of the basis vectors ${\bas{e}}_{1,2}$ in \eqref{eq:natural:cartesian:basis}. What made the basi vectors ${\bas{e}}_{1,2}$ so special, anyway? Nothing at all. 

What we've come to is that we the \emph{components} of a vector depend on the basis that we choose. In $\mathbb{R}^3$ we usually use the basis ${\bas{e}}_{1,2,3}$ where the vectors point respectively along the $\hat{x}$, $\hat{y}$, and $\hat{z}$ directions. It is kind of an ``obvious'' basis, though it completely depends on the $\hat{x}$, $\hat{y}$, and $\hat{z}$ directions having some intrinsic meaning. They often do not: we could set up our coordinate system however we wanted. In fact, \emph{nowhere} in our definition of a vector space did we even assume that a coordinate system exists!

Indeed, it's the other way around: a choice of basis vectors \emph{defines} a `coordinate system' rather than the other way around.\sidenote{Though it really is dangerous to think about a vector space as having coordinates. We will see why when we talk about vector bundles and manifolds.} All of this begs for a re-definition.





\section{Basis vectors, formally}

A \textbf{basis} is a \emph{minimal set} of vectors that span a space. Here `minimal' means that if you remove any vector from the basis, then there are vectors in the space that cannot be written as a linear combination of the remaining vectors.

\begin{example}\label{eg:over:specified:basis}
Consider the following three vectors:
\begin{align}
    \vec{v} &=
    \begin{pmatrix}
        1\\0\\0
    \end{pmatrix}
    &
    \vec{w} &=
    \begin{pmatrix}
        0\\1\\0
    \end{pmatrix}
    &
    \vec{u} &=
    \begin{pmatrix}
        1\\-1\\0
    \end{pmatrix} \ .
    \label{eq:tvu:example:basis}
\end{align}
These three vectors are \emph{not} a basis for a subspace because there are vectors that are linear combinations of $\vec{v}$, $\vec{w}$, and $\vec{u}$ that can be equivalently written as a linear combination of just $\vec{v}$ and $\vec{u}$, for example.
% 
To see this, consider the vector
\begin{align}
    \vec{t} = 4\vec{v} + 2\vec{w} + 3\vec{u} 
    =
    \begin{pmatrix}
        \pp 7 \\ -1 \\ \pp 0
    \end{pmatrix} \ .
\end{align}
This may equivalently be written as
\begin{align}
    \vec{t} = 7\vec{v} - \vec{w} \ .
\end{align}
Indeed, there are an infinite number of ways to write $\vec{t}$. Because $\vec{v} - \vec{w} + \vec{u} = 0$, you can add any multiple of this linear combination to $\vec{t}$ to leave $\vec{t}$ unchanged.
\end{example}

The \textbf{dimension} of a vector space is the number of vectors in the basis. In the example above, the vector space spanned by linear combinations of $\vec{v}$, $\vec{w}$, and $\vec{u}$ has dimension two. This is because you only need two vectors write any vector in the space as a linear combination. If you drew all of the vectors in this subspace as arrows with their base at the origin, then the arrow heads with all live on the $xy$-plane.

Here are some \emph{obvious} statements\sidenote{This means: if they are not immediately apparent, stop and think about it to make sure you understand.}:
\begin{enumerate}
    \item The zero vector cannot be part of any basis.
    \item The dimension of a vector space does not depend on the choice of basis.
    \item If you have a proposed set of basis vectors but there is a \emph{non-trivial} linear combination of those vectors that sums to zero, then the set of vectors is not a basis. Here non-trivial means ``the coefficients are not all equal to zero.'' This should be evident from Example~\ref{eg:over:specified:basis}.
    \item If any two vectors in a proposed basis are proportional to one another, then this set of vectors is not a basis.
    \item The number of components $v^i$ to describe a vector is the dimension of the vector space.
    \item In the expansion of a vector as a linear combination of basis vectors, the coefficients are unique to the vector. That is: if $\vec{v} = \sum_i \alpha^i\vec{b}_i$ for a basis $\vec{b}_{1,2,3}$, then the set of numbers $(\alpha^1, \alpha^2, \alpha^3)$ uniquely defines $\vec{v}$. There is no other combination of coefficients in a linear combination that sum to $\vec{v}$. 
    \item When describing a vector, the \emph{coefficients} of the linear combination of basis vectors and the \emph{components} of a column vector with respect to that basis are identical. This is by definition. 
\end{enumerate}
The last point is poignant. You may have believed that a vector \emph{is} the column of numbers. We want to move away from this so that we may generalize our definition. A vector is a linear combination of basis vectors, where we are remaining agnostic about what the basis vectors are. Let me say this again: \emph{the column of numbers is not the vector, it is simply a representation of a linear combination of basis vectors. All the ``vector-ness'' is encoded in the basis vectors.}

\begin{example}
Color space is a vector space that highlights this idea of a more abstract basis vector. In color theory, all colors are linear combinations of red, green, and blue. This should sound really weird because in physics these colors are simply wavelengths of light: what is special about them? Nothing in nature. What is special is that our eyes have three types of color receptor cells. Each type is sensitive to a certain window of the visible spectrum. We call these human eye responses the colors red, green, and blue. When we add colors, what we really mean is we're adding ``responses'' to a particular spectrum of light. When we add colors, we are not adding electromagnetic waves: we are adding neurological responses. For each type of color-sensitive cell, one `blip' of neural response is a basis vector for our color response. The sensation of a particular color is a linear combination of this basis. An actual human being is not sensitive to the whole vector space: for example, we cannot add negative colors to our sensory response. This is a fascinating subject and a surprising application of linear algebra.\footnote{There are some great YouTube videos on this. Here are a few: \url{https://www.youtube.com/watch?v=xAoljeRJ3lU}, \url{https://www.youtube.com/watch?v=AS1OHMW873s}, \url{https://www.youtube.com/watch?v=99v96TL-tuY}.}
\end{example}


What is less obvious is that at this point there is \emph{no preferred basis}. Any minimal set of vectors that span a vector space is a perfectly good vector. Suppose $\vec{b}_{1,2,3}$ is one such set for $\RR ^3$. We can write the vector with respect to the coefficients of the linear combination of  $\vec{b}_{1,2,3}$ basis vectors that reproduces it. If we had another basis, $\vec{b}'_{1,2,3}$, we could write the same vector with respect to a different linear combination of the $\vec{b}'_{1,2,3}$ basis. The components of these linear combinations, say $\alpha^i$ and $\alpha'^i$, will be different because the basis elements are different. However, they represent the same vector.

In your heart, you should feel anxious. You \emph{like} the ``obvious'' basis 
\begin{align}
    {\bas{e}}_1 &=
    \begin{pmatrix}
        1 \\ 0 \\ 0
    \end{pmatrix}
    &
    {\bas{e}}_2 &=
    \begin{pmatrix}
        0 \\ 1 \\ 0
    \end{pmatrix}
    &
    {\bas{e}}_2 &=
    \begin{pmatrix}
        0 \\ 0 \\ 1
    \end{pmatrix} \ .
\end{align}
We can even call this the Cartesian basis.
It seems so natural, we even gave these basis vectors little hats to remind us how much we like them! Stop and think about what you like about this basis. I guarantee you that all of those nice features invoke mathematical machinery that are \emph{not} included in a vector space. You may like that the Cartesian basis is orthogonal and each basis vector has unit length. To this I reply: \emph{how do you measure angle or length? These are not concepts that our vector space is equipped with.} You are, of course, correct that there \emph{is} a way to \emph{define} angle and length---but that is something additional that we have to impose on the space. We will get to this shortly.


\begin{example}\label{eg:polynomial:space}
Another surprising example of a vector space is the space of polynomials of finite degree. This means functions of the form
\begin{align}
    f(x) &= a^0 + a^1 x + a^2x^2 + \cdots a^N x^N \ .
\end{align}
Finite degree means that $N$ is some integer that is not infinity.\footnote{This may seem like a silly point, but one of the key `aha' ideas in this course will be that we can do linear algebra on the space of general functions where we allow $N\to \infty$.}
To be clear, our notation has become a bit ambiguous: here $x$ is a variable and $x^n$ means $x$ to the $n^\text{th}$ power. The coefficients $a^i$, on the other hand, are numbers and $i$ is an index. We can pick the following basis:
\begin{align}
    {\bas{e}}_0 &= x^0 = 1 &
    {\bas{e}}_1 &= x^1 = x &
    {\bas{e}}_2 &= x^2 &
    \cdots&&
    {\bas{e}}_N &= x^N &
\end{align}
These `basis vectors' are actually functions that are simple monomial powers of $x$. It should be obvious (there's that phrase again) that linear combinations of these basis vectors/functions can give any function $f(x)$ of degree up to $N$. It should also be obvious that the dimension of this space is $(N+1)$; don't forget to count the ${\bas{e}}_0$ vector.\par

For example, consider the polynomial $f(x) = 3+x^2$. The linear combination of basis vectors that gives this has $a^0 = 3$, $a^2 = 1$, and all other coefficients zero:
\begin{align}
    f(x) = 3{\bas{e}}_0 + {\bas{e}}_2 \ .
\end{align}
We could represent this vector/function as a column:
\begin{align}
    f(x) = \vec{f} = 
    \begin{pmatrix}
        3 \\
        0 \\
        1 \\
        0 \\
        \vdots  \\
        0
    \end{pmatrix}
\end{align}
where $\vec{f}$ is an $(N+1)$-component column of the numbers $a^i$.
\end{example}

\begin{exercise}
Consider a vector/function $\vec{f}$ with components $f^i$ in the polynomial space in Example~\ref{eg:polynomial:space}. Now consider the vector/function $\vec{f}' \equiv df/dx$. Write out an expression for the $i^\text{th}$ component of $\vec{f}'$. \emph{Hint}: for example, the $i=1$ component is $2f^2$.
\end{exercise}



\section{The meaning of column vectors}

The previous subsection on bases\footnote{The plural of `basis' is `bases,' pronounced \emph{bay-sees}, just like the plural of `axis' is `axes' pronounced \emph{axe-sees}.} is so important that we should really emphasize the mathematical edifice that we have reverse-engineered\footnote{Do you appreciate why I say `reverse engineered' here? In mathematics classes, one woud start with some postulates for what an abstract vector is and then your usual 3-component column vectors pop out as one silly example. We have started those 3-component column vectors and used their properties to motivate a general definition of what vectors are.}
\begin{enumerate}
    \item A vector is technically \emph{not} the column of numbers that we usually say it is. That column of numbers is simply a way of writing the \emph{components} of a vector.
    \item The components of a vector are simply the coefficients of the basis vectors in the linear combination of basis vectors that sum to the vector. That is: $v^i$ is defined by $\vec{v} = v^i {\bas{e}}_i$ where the ${\bas{e}}_i$ are a set of basis vectors that we all agree upon.
    \item I never had to say what the basis vectors \emph{are}. They can be anything where linear combinations of those things are still the same type of thing. In this way, we can treat the basis vectors abstractly.
\end{enumerate}
You may be used to vectors being forces, momenta, velocities, electric fields, and so forth. We want to be able to use the same machinery of linear algebra on more general objects: particles with quantum spin, functions, the perception of colors by the human eye, and so forth.


\begin{exercise}
What happens when we do not agree on a basis? Suppose you set up a basis. Stand up. Suppose you are facing north. Your feet are at the origin. If you spread your arms out, your right hand points in the direction of your first basis element ($x$-axis, pointing east), ${\bas{e}}_1$. Your nose points in the direction of your second basis element ($y$-axis, north), ${\bas{e}}_2$. Your head points in the direction of your third basis element ($z$-axis, skyward), ${\bas{e}}_3$.

However, your friend Riley approaches you from the northeast so Riley is facing southwest. Riley decides to set up their own basis, analogous to you. Their first basis element ${\bas{r}}_1$ points in the northwest direction, their second basis element ${\bas{e}}_2$ points in the southwest direction, and their third basis element ${\bas{e}}_3$ also points skyward. 

For simplicity, assume that the length of your basis vectors are all the same---even though we haven't defined what length means. Suppose you `measure' a vector with components $v^1 = 2$, $v^2=-1$, and $v^3=1.5$. This is a vector pointing southeast and upward. What components does Riley measure with respect to their basis?
\end{exercise}

\begin{exercise}
One of my favorite examples of a vector space is the space of Fibonacci sequences. Fibonacci sequences are infinite lists of numbers $a_i$ that satisfy $a_{i+2} = a_i+a_{i+1}$. Once you specify the first two numbers $a_0$ and $a_1$, you can iteratively generate every other number in the sequence. Each sequence is a vector in the space of possible Fibonacci sequences. Show that this is true by confirming that a linear combination of Fibonacci sequences with each $i^\text{th}$ term added, e.g.\ $(a+b)^i = a_i+b_i$ is also a Fibonacci sequence. Give an example of a basis for the Fibonacci sequences. What is the dimension of the Fibonacci sequence space? \emph{Answer}: the dimension is two, even though each element is an infinitely long list of numbers.
\end{exercise}

In some classes there is an agreed upon notion that the basis vectors are 
\begin{align}
    \bas{e}_1 &= \hat x_1 = \hat i
    &
    \bas{e}_2 &= \hat y_1 = \hat j
    &
    \bas{e}_3 &= \hat z_1 = \hat k \ .
\end{align}
This is not one of those classes. Often this \emph{canonical} basis is the obvious one to use, and we will try to use it. However, in this class you must be able to accept that we want to \emph{abstract} away the identity of the basis vectors. The basis vectors simply carry the vector-ness of a vector so that we can focus only on dealing with the \emph{components}, which we write as columns of numbers.




\section{Operations that are not (yet?) allowed}

In these definitions, we make a big deal about how the sum of two vectors \emph{is also a vector}. Or how the rescaling of a vector by a number \emph{is also a vector}. This is in contrast to operations that are either not allowed or that do not produce vectors. An example of an operation that is not allowed is adding together vectors from two different vector spaces. The following proposed sum of a vector in $\RR ^3$ with  a vector in $\RR ^2$ does not make sense:
\begin{align}
    \begin{pmatrix}
        v^1\\ v^2 \\v^3
    \end{pmatrix}
    +
    \begin{pmatrix}
        w^1\\ w^2 
    \end{pmatrix}
    =
    \; ?
\end{align}
If you find yourself adding vectors from two different vector spaces, then you have made a mistake.

Another operation that requires care is rescaling a real vector by a complex number. If $\vec{v}$ is a vector in $\RR ^3$ and we try to multiply it by a complex number, $\alpha = 2+3i$, then the resulting ``vector'' is \emph{not} a vector in $\RR^3$:
\begin{align}
    (\alpha\vec{v})^i = (2+3i)v^i \notin \RR  \ ,
\end{align}
that is: the components of $\alpha\vec{v}$ are not real numbers, and so this cannot be an element of aa vector space that is \emph{defined} to have real components. Later on we will generalize to the case of \emph{complex vector spaces}, but we will treat that with some care.\sidenote{If you want to be fancy, you can replace `number' with the mathematical notion of a \emph{field}\index{field}. Both the real numbers and the complex numbers are examples of fields. In my mind a field is just a class of number, though mathematicians have fancier definitions.}

Thus far, we have introduced the \emph{nouns} of this course: vectors. We have identified a few \emph{verbs} that let us do things with these vectors:
\begin{enumerate}
    \item Addition takes two vectors in a vector space and returns a vector in the same vector space. 
    \item Rescaling takes a vector and a number and returns a vector in the same vector space.
\end{enumerate}
We can rewrite this in the language of \emph{mappings} (or \emph{functions}) as follows. Let $V$ be a vector space, say $V=\RR ^3$. Let us write $\RR $ mean [real] numbers. Then the above statements tell us that addition and rescaling can be thought of as maps:
\begin{enumerate}
    \item Vector addition: $V\times V \to V$
    \item Rescaling: $V\times \RR  \to V$ \ .
\end{enumerate}
Do not be intimidated by the $\times$ symbol here. This ``mapping'' notation means nothing more and nothing less than the statements above.

We now know everything there is to know about the vector space $\RR ^3$. We want to learn more about functions (maps) that involve this vector space. How can we combine vectors and numbers to produce other vectors and numbers? What about more complicated objects like matrices and tensors? 

\section{Euclidean three-space}

You may object: \emph{wait! I know there are more things you can do with three-vectors!} You remember that there are two types of vector multiplication that we use in physics. The \textbf{dot product} and the \textbf{cross product}. 

In $\RR ^3$, the \textbf{dot product} is a map $V\times V \to \RR $. That means it takes two vectors and returns a number. The particular number that it returns is typically \emph{defined} to be
\begin{align}
    \vec{v} \cdot \vec{w} 
    = \sum_i v^i w^i  
    = v^1w^1 + v^2 w^2 + v^3w^3 \ .
    \label{eq:euclidean:3d:metric:intro}
\end{align}
The dot product generalizes in linear algebra. It is often called an \textbf{inner product} or a \textbf{metric} and has a few different notations that we will meet. What is important is that this dot/inner product is an \emph{additional} mathematical function that we attach to a vector space. 

Three-dimensional real space combined with the dot product/inner product/metric \eqref{eq:euclidean:3d:metric:intro} is called Euclidean three-space. In general, a vector space combined with a `dot product' is called a \textbf{metric space}. The word metric should invoke some etymological notion of measurement of distance. Indeed, the dot product is a tool that tells us how `close' two vectors are to one another---though it is not yet obvious how.

\begin{example}
Let $\mathbf{r}=(x,y,z)$ be a ``position vector'' of a point relative to the origin.\footnote{It is dangerous to use the phrase ``position vector,'' see Exercise~\ref{ex:position:vector}.} Then the distance of the point from the origin is
\begin{align}
    d = \sqrt{\vec{r}\cdot\vec{r}} =
    \sqrt{x^2+y^2 +z^2} \ .
    \label{eq:distance:in:space}
\end{align}
This gives a notion of how the dot product is related to measuring distances, but it turns out to be a bit of a red herring! The real sense in which the dot product measures the `closeness' of two vectors is the sense in which it defines an angle between those vectors. (See below.)
\end{example}

The \textbf{cross product} is a different story. You may remember the cross product from such hits as\footnote{\url{https://tvtropes.org/pmwiki/pmwiki.php/Main/YouMightRememberMeFrom}} angular momentum, $\vec{r}\times\vec{p}$. It looks like a map that takes two vectors and spits out another vector, $V\times V \to V$. Indeed, this is the case in Euclidean three-space. However, it had some funny properties compared to the dot product. For example, there was something weird with the order of the two vectors: $\vec{a}\times \vec{b}  = - \vec{b}\times \vec{a}$. It is also a bit funny that the direction of the output vector is completely different\sidenote{The technical meaning of `completely different' is \emph{orthogonal}, which we define below with the help of the metric.} from the directions of the input vectors. It will turn out that this product does not generalize as simply as the dot product, though there is a generalization called the \textbf{wedge product} which is outside the scope of this course.\sidenote{That is not to say that the wedge product is not relevant in phsyics. The wedge product features prominently in a mathematical field called \emph{differential geometry}, which is in turn the framework for general relativity. The wedge product is related to defining volumes and integration measures. You may also be interested in a field called \emph{geometric algebra} which is a related approach to the wedge product.\footnotemark}\footnotetext{e.g.\ see \url{https://www.youtube.com/watch?v=60z_hpEAtD8}}

\begin{exercise}
Define the generalization of the Euclidean three-space metric to Euclidean space in $d$ dimensions. (Easy.)
\end{exercise}

\begin{exercise}
Try to define a generalization of the cross product in two-dimensional Euclidean space. Reflect on why this is much less natural than the generalization of the dot product. 
\end{exercise}



\section{Length in Euclidean three-space}

Euclidean three-space is real space combined with the Euclidean dot product, \eqref{eq:euclidean:3d:metric:intro}. The [Euclidean] \textbf{magnitude} (length) of a three vector $\vec{v}$ as $|\vec{v}|$ in Euclidean three-space. The magnitude is defined to be
\begin{align}
    |\vec{v}| = \sqrt{\vec{v}\cdot\vec{v}} \ .
\end{align}
This definition generalizes to Euclidean $d$-dimensional space with the appropriate generalization of the dot product.

\begin{example}
Consider the vector
\begin{align}
    \vec{v} = 
    \begin{pmatrix}
    \phantom{+}3\\-4\\\phantom{+}0    
    \end{pmatrix}
\end{align}
in Euclidean three-space. The magnitude of $\vec{v}$ is $|\vec{v}| = 5$.
\end{example}


Some references prefer to use the double bar notation, $||\vec{v}||$ for the length of a vector. This is to distinguish it from the absolute value of a number, $|-3| = 3$. We will be even more perverse: sometimes we will write $v$ to mean the magnitude of $\vec{v}$ when there is no ambiguity.

\begin{example}
Consider the vector
\begin{align}
    \vec{v} = 
    \begin{pmatrix}
    -1\\ \phantom{+}3\\ \phantom{+}2
    \end{pmatrix} \ .
\end{align}
Then the \emph{magnitude} of $\vec{v}$ is $|\vec{v}|=\sqrt{14}$. We could also write this as $v = \sqrt{14}$, but we should be careful when we write things like $v^2$ which could either mean the second component of $\vec{v}$---which is $3$---or the square of the magnitude---which is 14. 
\end{example}


We see that the dot product (metric) allows us to define length. Because the length of a vector is a number, we can divide the vector $\vec{v}$ by its its length $|\vec{v}|$ to obtain a \textbf{unit vector}:
\begin{align}
    {\bas{v}} = \frac{1}{|\vec{v}|}\vec{v} \ .
    \label{eq:eg:v:340}
\end{align}
The right-hand side is simply scalar multiplication by $|\vec{v}|\inv$. Unit vectors are useful for identifying directions.

\begin{example}
In grade school one may have learned that a vector is an arrow that has a magnitude and a direction. Unit vectors encode the `direction' of a vector.
\end{example}

\begin{example}
Let $\vec{v}$ be defined as in \eqref{eq:eg:v:340}. The unit vector associated with $\vec{v}$ is
\begin{align}
    \hat{v} = 
    \begin{pmatrix}
        \phantom{+}3/5 \\
        -4/5\\
        0
    \end{pmatrix} \ .
\end{align}

\end{example}

\section{Matrix Multiplication}
\label{sec:matrix:multiplication}

A matrix is an object that acts on vectors to give you other vectors. In this way, we say that it encodes a \emph{transformation} of a vector. For simplicity, let us write this out for two-component vectors. Matrices are then $2\times2$ arrays that look like this:
\begin{align}
    M = \begin{pmatrix}
        a & b\\
        c & d
    \end{pmatrix} \ .
\end{align}
The numbers $a$, $b$, $c$, and $d$ are the components of the matrix. These act on a vector, which we shall take to be
\begin{align}
    \vec{v} = \begin{pmatrix}
        x \\ y
    \end{pmatrix} \ .
\end{align}
% 
When we multiply $\vec{v}$ by the matrix $M$ we get a new \emph{vector} which we write $M\vec{v}$. 
There is a silly rule to tell us what the components of the $M\vec{v}$ vector are.
The rule is visualized in Figure~\ref{fig:matrix:mult:2220}.
% \begin{figure}[ht]
%     % \centering
%     \sidecaption[][-2\baselineskip]{%
%         Matrix multiplication rule visualized. 
%         %
%         %% \label command inside the \sidecaption command
%         \label{fig:matrix:mult:2220}
%     }
%     \includegraphics[width=.5\textwidth]{figures/MatrixMult_2220.pdf}
% \end{figure}
\begin{marginfigure}%[th]
    \includegraphics[width=\textwidth]{figures/MatrixMult_2220.pdf}
    \captionsetup{font={scriptsize,sf}}
    \caption{Multiplying a vector by a matrix.}
    \label{fig:matrix:mult:2220}
\end{marginfigure}
It is clunky to say the rule in words, but it goes something like this: 
\begin{enumerate}
    \item The components in the \emph{first} row of the matrix combine with the components of the column of the vector to give the \emph{first} component of $M\vec{v}$. To enact this visually,  highlight the \emph{first} row and rotate the array of numbers in $M$ clockwise by 90 degrees. The components of $M$-tipped over that are the same height as the corresponding components of $\vec{v}$ are multiplied and each product is summed together. This sum is the \emph{first} component of $M\vec{v}$.
    \item The components in the \emph{second} row of the matrix combine with the components of the column of the vector to give the \emph{second} component of $M\vec{v}$. To enact this visually,  highlight the \emph{second} row and rotate the array of numbers in $M$ clockwise by 90 degrees. The highlighted components of $M$-tipped over that are the same height as the corresponding components of $\vec{v}$ are multiplied and each product is summed together. This sum is the \emph{second} component of $M\vec{v}$.
\end{enumerate}
If you are working with three-component vectors, then there is a third step where the word `second' is replaced by `third.' 

There are other kinds of objects called row vectors. These look like vectors but they have been tipped over counter-clockwise by 90 degrees:
\begin{align}
    \row{w} = \begin{pmatrix}
        a & b
    \end{pmatrix} \ .
\end{align}
We can use this same `tip over and multiply same-height components' visualization to multiply row vectors onto vectors. See Figure~\ref{fig:matrix:mult:0220}.
\begin{figure}[ht]
    \centering
    \captionsetup{font={scriptsize,sf}}
    \sidecaption[][-2\baselineskip]{%
        Matrix multiplication of a row vector onto a [column] vector.  
        %
        %% \label command inside the \sidecaption command
        \label{fig:matrix:mult:0220}
    }
    \includegraphics[width=.8\textwidth]{figures/MatrixMult_0220.pdf}
\end{figure}
Observe that the result of this multiplication is a number.
\begin{exercise}
Draw the vector
\begin{align}
    \vec{v} = \begin{pmatrix}
        1 \\ 1
    \end{pmatrix} \ .
\end{align}
Draw the vector
\begin{align}
    \begin{pmatrix}
        2 & 1 \\
        1 & 1
    \end{pmatrix}
    \begin{pmatrix}
        1 \\ 1
    \end{pmatrix} \ .
\end{align}
\end{exercise}

We can also multiply matrices with one another. The result of this is another matrix. One can construct the components of this matrix by thinking of the left matrix acting on each column of the right matrix to give the corresponding column of the matrix product. Here's how one would find the top-left component of the product of two $2\times 2$ matrices:
\begin{figure}[ht]
    \centering
    \captionsetup{font={scriptsize,sf}}
    \sidecaption[][-2\baselineskip]{%
        Multiplication of two matrices, highlighting the steps to find the top-left component of the product matrix.
        %
        %% \label command inside the \sidecaption command
        \label{fig:matrix:mult:2222a}
    }
    \includegraphics[width=.8\textwidth]{figures/MatrixMult_2222a.pdf}
\end{figure}
We can go on and find the top right (first row, second column) and bottom left (second row, first column) components of the product matrix, see Figure~\ref{fig:matrix:mult:2222b}.
\begin{figure}[ht]
    \centering
    \captionsetup{font={scriptsize,sf}}
    \sidecaption[][-2\baselineskip]{%
        Multiplication of two matrices, highlighting the steps to find the top right and bottom left components of the product matrix.
        %
        %% \label command inside the \sidecaption command
        \label{fig:matrix:mult:2222b}
    }
    \includegraphics[width=.8\textwidth]{figures/MatrixMult_2222b.pdf}
\end{figure}
\begin{exercise}
Show that the product of two $2\times 2$ matrices $MN$ is different from the product in the opposite order, $NM$. We say that matrix multiplication is not commutative.  
\end{exercise}
\begin{exercise}
Show that
\begin{align}
\begin{pmatrix}
    2 & 1 \\
    1 & 1 
\end{pmatrix}
\begin{pmatrix}
    1 & -1 \\
    -1 & 2 
\end{pmatrix}
=
\begin{pmatrix}
    1 & 0 \\
    0 & 1
\end{pmatrix} \ .
\end{align}
The matrix on the right-hand side is called the \textbf{identity matrix}, $\one$, because when it acts on a vector it leaves the vector unchanged. We say that the two matrices on the left-hand side are \textbf{inverses}. Show further that if two matrices are inverses, $M$ and $M\inv$, then the order of the multiplication does not matter: $MM\inv M\inv M = \one$.
\end{exercise}

One can further generalize this to non-square matrices---that is, matrices with a different number of rows than columns. Those matrices will not be of direct use in this course. However, the rules of matrix multiplication follow.
\begin{exercise}
Show that you can use the matrix multiplication rules multiply a $2\times 3$ matrix onto a $3\times 2$ matrix, where our notation is $(\textnormal{number of rows})\times(\textnormal{number of columns})$.
\end{exercise}
\begin{exercise}
Show that you \emph{cannot} multiply a $2\times 3$ matrix onto a $2 \times 3$ matrix. 
\end{exercise}
\begin{exercise}
Suppose you want to multiply an $n\times m$ matrix onto a $k \times \ell$ matrix.. What are the conditions on the numbers $n, m, k, \ell$ for this to make sense using the matrix multiplication rule?
\end{exercise}

This notion of matrix multiplication is helpful and perhaps something you may have learned in earlier stages of your education. It is still the way I do many calculations. \emph{However}, the rules in this section are a \emph{shortcut} for a much richer mathematical structure. It is this mathematical structure that we want to reveal because it shows us how seemingly different mathematical structures in physics are actually rooted in the same underlying language. As such, many of the notions in this chapter may be ideas that you must first \emph{unlearn} in order to \emph{relearn} how they are outputs of the richer structure.\sidenote{When I teach this class there are often a few students who are apologetic for not having taken a formal linear algebra class. I have noticed that those students sometimes do much better in the course because they have fewer preconceptions to unlearn.}


\chapter{Indexology}
\begin{quote}
... should not prevent us from avoiding purely formal calculations where a \emph{debauchery of indices} hides an often simple geometrical reality. -- E.~Cartan, \emph{Lecons sur la Geometrie des Espaces de Riemann}\sidenote{From Spivak, \emph{A Comprehensive Introduction to Differential Geometry}, volume 2.}\footnote{\cite{spivak1975comprehensive}; see also \url{spivak1975comprehensive}}
\end{quote} % cite https://hsm.stackexchange.com/questions/3320/the-debauch-of-indices-translation-request

In this chapter we introduce the rules for index notation. Please accept them for now as a necessary complication---there is nothing deep here, only a set of conventions and one useful shorthand (summation convention). The utility of these may not be obvious until we start to see how this is used and where it comes from.

\section{Tensors and index notation}
\label{sec:index:notation}

In this course we make a \emph{big deal} about the height of indices. This means the index does more than  `index' a number in an array. In fact, in physics there are objects that carry multiple indices with different heights. Here is one such example, the Riemann tensor in general relativity: $R^a_{\phantom{a}bcd}$. This object has four indices. The first one, $^a$ is raised, and the following three, $_{bcd}$ are lowered. There needs to be an unambiguous ordering of the indices: it is clear that the index $b$ is the \emph{second} index, the index $c$ is the \emph{third} index, and so forth. So it would be disastrous to write $R^a_{bcd}$ because now we cannot tell whether the upper index $^a$ or the lower index $_b$ is the \emph{first} index. This is demonstrated in Figure~\ref{fig:Riemann:tensor:for:indices}.
\begin{marginfigure}%[th]
    \includegraphics[width=.8\textwidth]{figures/Rabcd_eg.pdf}
    \captionsetup{font={scriptsize,sf}}
    \caption{The Riemann tensor showing the significance of the ordering and height of its indices.}
    \label{fig:Riemann:tensor:for:indices}
\end{marginfigure}

So let us get to the elephant in the room. To a physicist, a \textbf{tensor}\index{tensor} is an object that has indices. In this sense, vectors and matrices are both types of tensors. They can have any number of indices, but the indices have a well defined order and a well defined height (raised or lowered). In general, the following two objects are different:
\begin{align}
    M\aij{i}{j} &&\text{and} && M_i^{\phantom{i}j} \,
\end{align}
even though both are two-index objects whose first index is $i$ and second index is $j$. 

Why do we make such a big deal about indices and their heights? The difference between a tensor and an array of numbers is that tensors have specific \emph{transformation rules} under symmetries. The symmetry that you are most familiar with is rotational symmetry. From your first-year coursework, you are familiar with how useful it is to rotate to coordinates where a problem is simpler. The most common example of this is calculating the moment of inertia of a rotating body. There we had an object called the \emph{moment of inertia tensor}\sidenote{The transformation rules of this tensor are precisely why it is not called a ``moment of inertia \emph{matrix}.'' Though you may be hard pressed to find an honest textbook that explains this.\footnotemark}\footnotetext{See e.g.\,\url{https://hepweb.ucsd.edu/ph110b/110b_notes/node24.html}} One of the groan-inducing exercises in mechanics is to find the rotation in which the moment of inertia tensor of a rotating body is diagonal.

Here is what we need to know for now:
\begin{enumerate}
    \item In physics, tensors are objects with indices. These are arrays of numbers so that a particular choice of indices corresponds to a number in the tensor. The order of the indices matters.
    \item But there is more: whether an index is upper or lower indicates how that part of the tensor transforms under a symmetry transformation such a rotations. 
\end{enumerate}
At this point, you may have several questions, such as these:
\begin{enumerate}
    \item How exactly does a tensor transform under symmetries?
    \item What are examples of other symmetries?
    \item How should I visualize an object with more than two indices? (Yes, you can think of a three-index object as a hypercube arrays of numbers. No, I do not know of a good way to visualize upper versus lower indices on this array.)
\end{enumerate}
We shall answer these as we build up the machinery below. Just take this section as a request to believe that there may be method to this madness.

\section{The treachery of indices}

There is something that physicists do that tend to drive mathematicians crazy: we write a generic \emph{component of a vector} and refer to it as if it were the vector itself. It is a fairly harmless peccadillo:\sidenote{There are times when you can get into trouble if you drink your own Kool Aid, so to speak. The reason is that the \emph{component} $v^i$ is simply a number, whereas $\vec{v}$ is a vector. Some manipulations are only allowed for numbers and not vectors, and you should be clear that you mean `the component $v^i$' if you are treating it like a number, and not `the \emph{vector} whose components are $v^i$.' %See Example~\ref{eg:moving:coefficients:around}.
} if I say
\begin{quote}
the vector $v^i$
\end{quote}
then it is not hard to guess that I mean
\begin{quote}
the vector $\vec{v}$ which has components that I label $v^i$.
\end{quote}
If you ever meet a mathematician who gives you a hard time about this, you can wave your hands and refer to something called \emph{abstract index notation}, developed by Roger Penrose.\footnote{\url{https://math.stackexchange.com/questions/455478/}} To the best of my understanding, this is simply a formal way to justify the way physicists talk about indices. 

The reason why we have this culture is that this index notation ends up being so damn convenient. In addition to vectors, we will have other objects that have indices: dual vectors, matrices, and tensors. When we write everything in with indices, we can ``see'' properties of these objects that are not obvious without the indices. Specifically, we can see \emph{how an object transforms under symmetries}. In this course, we will focus on \emph{rotations} of vectors and their generalizations. 

Whenever I think about whether $v^i$ means the vector $\vec{v}$ or the $i^\textnormal{th}$ component of that vector, I am reminded of Magritte's ``The Treachery of Images,'' Figure~\ref{fig:Magritte}\footnote{From \url{https://en.wikipedia.org/wiki/The_Treachery_of_Images}, please refer to this page for fair use justification.}
\begin{marginfigure}%[th]
    \includegraphics[width=.8\textwidth]{figures/MagrittePipe.jpg}
    \captionsetup{font={scriptsize,sf}}
    \caption{``La Trahison des Images'' (``The Treachery of Images'') by Ren\'e Magritte. Owned by \tacro{LACMA}, reproduced here under fair use.}
    \label{fig:Magritte}
\end{marginfigure}
In this image, Magritte shows a painting of a pipe and then writes ``this is not a pipe.'' The implied message is that it is a \emph{painting} of a pipe that we may use to express the \emph{idea} of a pipe.


\section{Summation Convention}
\label{sec:summation}


There is another reason why indices are convenient: they allow us to use \textbf{summation convention}.\sidenote{Sometimes called Einstein summation convention in deference to its progenitor. With respect to Einstein, we simply write \emph{summation convention} because it's not like the dude is underappreciated in popular culture.} This is a notational shortcut that introduces upper and lower indices to convey sums. Consider, for example, the ``matrix multiplication'' of a row vector $\row{w}$ on a column vector $\vec{v}$. Nevermind the formal definition of ``row vector'' as opposed to ``column vector.'' Let us write it out in components where it is obvious for $\RR ^3$:
\begin{wide}
\begin{align}
    \row{w}
    &=
    \begin{pmatrix}
        w_1 & w_2 & w_3
    \end{pmatrix}
    &
    \vec{v}
    &=
    \begin{pmatrix}
        v^1 \\ v^2 \\ v^3
    \end{pmatrix}
    &
    \row{w}\vec{v}
    &= w_1v^1 + w_2v^2+w_3v^3 \ .
\end{align}
\end{wide}
On the far right we have used the matrix multiplication rules in Section~\ref{sec:matrix:multiplication}.
% The final expression is familiar, right? It follows the usual rules of matrix multiplication for a ``matrix'' that happens to be one row and three columns; 
We review this rule in Fig.~\ref{fig:row:col:mult}, labeling the components with upper and lower indices as appropriate.
% %% FIGURE SNIPPIT
\begin{marginfigure}%[tb]
    \centering
    \captionsetup{font={scriptsize,sf}}
    \includegraphics[width=\textwidth]{figures/rowcolmult.pdf}
    \caption{The `matrix multiplication rule' for acting with a row vector on a column vector.}
    \label{fig:row:col:mult}
\end{marginfigure}
We notice that we chose to write the components of $\row{w}$ with lower indices---this is the convention. Row vectors (which have many names) have indices written as subscripts while column vectors have indices written as superscripts. There is no mathematics here, just a choice of notation. The result of the multiplication is simply a number, which we can write as a sum:
% \begin{align}
%     \row{w}\vec{v}
%     &= \sum_{i=1}^3 w_iv^i
%     \equiv w_iv^i \ .
%     \label{eq:row:w:on:vec:v}
% \end{align}
% On the right-hand side we have \emph{defined} the summation convention: \emph{whenever there is exactly one upper index and exactly one lower index with the same letter, we should understand that there is a sum over that index over all of its allowed values.} We call pairs of repeated indices where one is upper and one is lower \textbf{contracted indices}.


% The value $w_iv^i$ is simply a number. It is not a vector. It does not have any ``vectorial'' (tensorial) structure. It is not an element of the vector space $\RR ^3$. It does not transform under rotations. It is \emph{just a number}. In other words, $w_iv^i$ behaves like an object with \emph{no indices}. Contracted indices ``cancel each other out.''

% This is significant because we will see that indices tell us how objects transform. Evidently, column vectors and row vectors transform differently since one has an upper index and one has a lower index. Further, when we contract the two indices, we end up with something with no indices: a number that does not transform at all. This may seem like notational overkill---trust me, it is worth building this notation now. We will use it over and over.

% \begin{figure}[tb]
%     \centering
%     \includegraphics[width=.5\textwidth]{figures/matrixmultiplication.pdf}
%     \caption{The `matrix multiplication' rule for $A\vec{v} = \vec{v}'$. We show that the second element of $\vec{v}'$ is a sum of terms, where each term is a multiplication of the $j^\text{th}$ column of the $2^\text{nd}$ row of $A$ by the $j^\text{th}$ row of $\vec{v}$.}
%     \label{fig:matrix:col:mult}
% \end{figure}

% \begin{example}
% Matrices $M$ have the following index structure: $M\aij{i}{j}$. There is a first index and a second index---the order matters. The first index is upper, and the second index is lower. Matrix multiplication boils down to a contraction of indices:
% \begin{align}
%     (M\vec{v})^i = M\aij{i}{j}v^j \ .
%     \label{eq:matrix:mult:ith:comp}
% \end{align}
% Let us read this equation carefully. First, $M\vec{v}$ is a vector. The $i^\text{th}$ component of this vector is $(M\vec{v})^i$. How is this related to the components of $M$ and $\vec{v}$? The right-hand side tells us that we simply take the sum:
% \begin{align}
%     M\aij{i}{j}v^j = 
%     M\aij{i}{1}v^1 + M\aij{i}{2}v^2  + M\aij{i}{3}v^3 \ .
% \end{align}
% \end{example}
% \begin{example}
% From the above example, you can then excuse the glib statement: ``the \emph{vector} $M\aij{i}{j}v^j$.'' As we explained above, $M\aij{i}{j}v^j$ is not a vector, but a component of a vector. However, the point is that even though there are three indices, two of them are contracted so the object effectively only has one upper index. This is the index structure of a vector. This matches the usual matrix multiplication rule shown in Fig.~\ref{fig:matrix:col:mult}.
% \end{example}

% \begin{exercise}
% Consider the following vector, row vector, and matrix:
% \begin{align}
%     \vec{v} &=
%     \begin{pmatrix}
%      1 \\ 2 \\ 3   
%     \end{pmatrix}
%     &
%     \row{w} &=
%     \begin{pmatrix}
%         4&5&6
%     \end{pmatrix}
%     &
%     M&=
%     \begin{pmatrix}
%         1 & 2 & 3 \\
%         4 & 5 & 6 \\
%         7 & 8 & 9
%     \end{pmatrix} \ .
% \end{align}
% These have index structure $v^i$, $w_i$, and $M\aij{i}{j}$ respectively. Note that the first index of a matrix is the row and the second is the column, thus $M\aij{1}{2} = 2$ while $M\aij{2}{1} = 4$. Calculate the following: $(wM)_2$, $(Mv)^1$, $(MM)\aij{1}{2}$. Here $MM$ is understood to be the square of the matrix $M$, $(M^2)\aij{i}{j} = M\aij{i}{k}M\aij{k}{j}$.
% \end{exercise}

% \begin{example}\label{eg:moving:coefficients:around}
% It should be clear that
% \begin{align}
%     w_i M\aij{i}{j} = 
%     w_1 M\aij{1}{j} + w_2 M\aij{2}{j} + w_3 M\aij{3}{j}
%     = 
%     M\aij{1}{j}w_1  + M\aij{2}{j}w_2 + M\aij{2}{j}w_2
%     =
%     M\aij{i}{j}w_i \ .
% \end{align}
% After all, each of the components $w_i$ and $M\aij{i}{j}$ are simply numbers. However: even though $w_i M\aij{i}{j} = M\aij{i}{j}w_i$, it is \emph{completely incorrect} to say $\row{w}M = M\row{w}$. This is because $\row{w}$ and $M$ are \emph{tensorial} (vector-y) objects. The order of their `multiplication' matters. You can see this from the matrix notation.
% \begin{align}
%     \row{w}M &= 
%     \begin{pmatrix}
%         4&5&6
%     \end{pmatrix}
%     \begin{pmatrix}
%         1 & 2 & 3 \\
%         4 & 5 & 6 \\
%         7 & 8 & 9
%     \end{pmatrix}
%     &
%     M\row{w} &=
%     \begin{pmatrix}
%         1 & 2 & 3 \\
%         4 & 5 & 6 \\
%         7 & 8 & 9
%     \end{pmatrix}
%     \begin{pmatrix}
%         4&5&6
%     \end{pmatrix} \ .
% \end{align}
% The first multiplication gives a row vector, as you expect since $(wM)_j$ has one lower index. The second multiplication does not even make sense. What we see is that expressions like $w_i M\aij{i}{j} = M\aij{i}{j}w_i$ are valid as long as you are only talking about the components. The glib ``physicist slang'' of replacing a component by its vector/matrix/tensor can get you into trouble if you have moved components around in a way that is only allowed for numbers, but not vectory-things.
% \end{example}

% Since the language is now becoming cumbersome, let us define the word \textbf{tensorial} to mean an object with indices. This will replace the phrase ``vectory'' in our notes.




% One neat thing about this is that our convention for contracting indices makes it clear that $(Mv)^i$ is a component of a vector: it has one upper index. Similarly, you may recall that the multiplication of matrices $M$ and $N$ proceeds as follows:
% \begin{align}
%  (MN)\aij{i}{j} = M\aij{i}{k}N\aij{k}{j} \ .
%  \label{eq:matrix:matrix:multiplication}    
% \end{align}
% \begin{exercise}
% Confirm that \eqref{eq:matrix:matrix:multiplication} holds for $2\times 2$ matrices.
% \end{exercise}
% On the right-hand side of \eqref{eq:matrix:matrix:multiplication}, we have one pair of contracted indices ($k$), one upper index $i$, and one lower index $j$. We thus deduce that this object is a matrix: it has one upper and one lower index. Indeed, the product of two matrices is also a matrix. Our indices and contraction rules tell us what kinds of objects we can produce by contracting indices between them. 

% \begin{example}
% You may also contract indices within an object. For example, because a matrix has one upper and one lower index, you may contract them together. This is called the trace, $\Tr M = M\aij{i}{i}$.
% \end{example}


















\chapter*{Closing Thoughts}

\section*{Acknowledgments}

\acro{PT}\ thanks the students of Physics 17 (Spring 2022, 2023, and 2024) for their feedback and patience and thanks Owen Long for his support for creating this course. He also thanks Yakov Eliashberg, from whom he first heard the phrase ``\emph{there is no such thing as a position vector}.''
%
% \acro{PT} is supported by the \acro{DOE} grant \acro{DE-SC}/0008541.
\acro{PT} is supported by a \acro{NSF CAREER} award (\#2045333).

%% Appendices
\appendix
% \chapter{Proper appendix}
% Unlike Chapter~\ref{sec:subappendix:eg}, this is an appendix at the end of the document rather than a sub-appendix within a chapter. Check out the index that follows this chapter.

% \section{Things to work on}

% It may be nice to incorporate something like \texttt{classicthesis}\footnote{\url{https://www.ctan.org/tex-archive/macros/latex/contrib/classicthesis/}}


\printindex

%% Bibliography
%% USING BIBLATEX, SKIP BIBTEX
%% Use inspireHEP bibtex entries when possible
% \bibliographystyle{utcaps} 	% arXiv hyperlinks, preserves caps in title
% \bibliography{bib title without .bib}


\end{document}