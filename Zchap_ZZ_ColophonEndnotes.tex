\chapter{Colophon and Endnotes}


\section{End notes}

I end up doing a lot of things accidentally. There are a few things that I have chosen to do deliberately in these notes:
\begin{itemize}
	\item Large margins are invitations for readers to jot down notes as they read. Typographically, this makes text a bit easier to read\sidenote{This is why people liked two column environments when reading articles on paper.} without forcing readers on a screen to scroll up an down.\sidenote{This is why \emph{fewer} people like two column environments today.}
	\item The margins are also a place me to interject my own running \emph{Pale Fire}-esque narrative. Side notes are a way to add then flavor of being in a classroom: it is a place for parenthetical remarks, unbalanced opinions, and occasional reassurance. 
	\item Footnotes are primarily for references. Rather than a traditional bibliography for a textbook, lecture notes are an invitation to dive into a subject. I want references to be available on the page that they are referenced because they are \emph{not} there to point to earlier work, they are resources that I found helpful and want readers to also have at their finger tips.\sidenote{Having to flip to the back of a book or---even worse---scroll to the end of a pdf really kills one's momentum. It is much easier to have the references available at a peek at the bottom of the page. Kudos to pdf reader/writers like \textsc{Notability} for having easy functions to jump back and forth between document pages.}
\end{itemize}

\section{Colophon}

This document is prepared using the Flip Paper 2022 template\footnote{\url{https://github.com/fliptanedo/paper-template-2022}} and \acro{Sublime Text 4} on mac\acro{OS}. This particular manifestation is inspired by Edward Tufte's vision for typography and visual layout.\autocite{tufte2001visual} The document is typeset using typeset using the \LaTeX typesetting system originally developed by Leslie Lamport, based on \TeX created by Donald Knuth. 

