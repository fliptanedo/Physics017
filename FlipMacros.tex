%!TEX root = paper.tex
%% FLIP’S MACROS 
%% USES: FlipPreamble.tex

%% FOR `NOT SHOUTING' CAPS (e.g. acronyms)
%% ---------------------------------------
% \usepackage{scalefnt} 
% \newcommand\acro[1]{{\footnotesize #1}} 
\newcommand\acro[1]{{\small{#1}}} 
\newcommand\tacro[1]{\textsc{\lowercase{#1}}}  % for inside headers

%% COMMON PHYSICS MACROS
%% ---------------------
\renewcommand{\tilde}{\widetilde}         % tilde over characters
\renewcommand{\text}{\textnormal}	        % text in equations 
\renewcommand{\vec}[1]{\mathbf{#1}}       % vectors: boldface
\newcommand{\bas}[1]{\hat{\mathbf{#1}}}   % basis vectors: hat
\newcommand{\RR}{\mathbbm{R}}
\newcommand{\CC}{\mathbbm{C}}

\newcommand{\abs}[1]{\left\lvert#1\right\rvert}

%% LINEAR ALGEBRA
%% ---------------
\newcommand{\row}[1]{\tilde{\mathbf{#1}}} % row vectors have tilde

% have to compile from CTAN ("latex undertilde.ins")
\usepackage{undertilde} 
\renewcommand{\row}[1]{\utilde{\mathbf{#1}}}   % row vectors 

\newcommand{\rbas}[1]{\row{#1}}           % row basis

%% Have to compile ``latex undertilde.ins'' for this: 
% \usepackage{undertilde} 
%   % have to compile from CTAN ("latex undertilde.ins")
%   \renewcommand{\row}[1]{\utilde{\mathbf{#1}}}   
%   \renewcommand{\rbas}[1]{\hat{\row{#1}}}

\newcommand{\ket}[1]{\left|#1\right\rangle}       % <#1|
\newcommand{\bra}[1]{\left\langle#1\right|}       % |#1>

\newcommand{\aij}[2]{^{#1}_{\phantom{#1}#2}}
\newcommand{\mat}[3]{#1\aij{#2}{#3}}
\newcommand{\inv}{^{-1}}

\newcommand{\one}{\mathbbm{1}}
\newcommand{\Tr}{\text{Tr}\,}
\newcommand*{\trans}{{\mkern-1.5mu\mathsf{T}}}     % transpose



%% DIFFERENTIALS
%% -------------
%% Differential and differential/2pi
% \newcommand{\dbar}{d\mkern-6mu\mathchar'26}     % for d/2pi
\newcommand{\dbar}{d\mkern-6mu\mathchar'26\hspace{-.1em}}    

%% Best practice: Roman differential
\newcommand{\D}[1]{\ensuremath{\operatorname{d}\!{#1}}}
\newcommand{\DD}[2]{\ensuremath{\operatorname{d}^{#1}\!{#2}}}
\newcommand{\Dbar}[1]{\operatorname{d}\mkern-10mu\mathchar'26\mkern-2mu{#1}} 


%% TYPOGRAPHY: Best Practices
%% --------------------------
%% base of natural log is Roman
\newcommand{\e}{\operatorname{e}}  

%% imaginary number is Roman too!?
\newcommand{\I}{\operatorname{i}\mkern-2mu}  

%% phantom + for spacing (aligning in math environment)
\newcommand{\pp}{\phantom{+}}                     

%% Subscript parallel is same size as subscript perp
\usepackage{scalerel} % https://tex.stackexchange.com/a/523873/8094
\newcommand*{\paral}{{\stretchrel*{\parallel}{\perp}}}

%% For := with the dots and lines aligned, same size
%%% h/t tex.stackexchange.com/a/4881/8094
\newcommand*{\defeq}{\mathrel{\vcenter{\baselineskip0.5ex \lineskiplimit0pt
                     \hbox{\scriptsize.}\hbox{\scriptsize.}}}%
                     =}




%% Make my life easer
%% ------------------
\newcommand{\la}{\langle}
\newcommand{\ra}{\rangle}
\newcommand*{\smallslot}{\,\underline{\makebox[0.80em]{\ensuremath{}}}\,}
%   e.g. writing dual vector as <_,v> 



%% MISCELLANEOUS
%% -------------
\usepackage{pifont}
  \newcommand{\cmark}{\ding{51}}%
  \newcommand{\xmark}{\ding{55}}%

