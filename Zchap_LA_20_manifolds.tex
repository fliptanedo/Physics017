\chapter{Calculus on Manifolds}


\section{The central confusion}

Let us return to the point of Chapter~\ref{ch:bundles}, a point that we heave repeated as ``there is no such thing as a position vector.'' A common source of confusion is that in flat space, we typically confuse two very different kinds of spaces:
\begin{itemize}
	\item Manifolds: this is what physicists usually call a \emph{space} or \emph{spacetime}. For our purposes, physical objects live their lives on spacetime. Those objects can have trajectories on the manifold. Manifolds may be curved.
	\item Tangent spaces: these are vector spaces that live on each point on a manifold. Vectors live in these spaces. We can define these vectors to be the velocities of possible trajectories through that point on the manifold. Each tangent space is independent and one has to define how to connect the tangent spaces of neighboring points. The nature of this connection is what we mean by a manifold being curved.
\end{itemize}
\begin{example}
Spaces and spacetimes are common manifolds in physics. However, we can also have more abstract manifolds. In statistical mechanics, \emph{phase space} is the manifold of positions and momenta (or velocities) for every particle in a system. A system with one particle has a six dimensional phase space: three positions and three momenta. A system with ten particles has a $6\times 10$-dimensional phase space. 
\end{example}


\section{Polar Coordinates}

Coordinates can be curvy, even when the spacetime is flat. Tangent spaces---where linear algebra lives---are `regular' vector spaces where we like to work with orthonormal basis vectors.

\section{Calculus Review}

\flip{Work in progress}